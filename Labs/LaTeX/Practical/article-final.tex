% make sure you have the VPN on, so that latex can load packages on the fly


\documentclass{article}

% graphics package
\usepackage{graphicx} 

% enhanced citation package 
\usepackage{natbib}
\bibpunct{(}{)}{;}{a}{}{,}  % to adjust punctuation in references


% adjust caption properties
\usepackage[margin=10pt, font=small, labelfont=bf]{caption} 

% hyperrefs on, with nicer colors
\usepackage{color}
\usepackage{xcolor}
\usepackage[]{hyperref}
\definecolor{darkblue}{rgb}{0,0,.5}
\hypersetup{colorlinks=true, breaklinks=true, linkcolor=darkblue, menucolor=darkblue, urlcolor=darkblue, citecolor=darkblue}

% enhanced tables
\usepackage{multicol}              
\usepackage{multirow}
\usepackage{booktabs}  


\author{Florian Hartig}
\title{My first \LaTeX document}

\begin{document}
\maketitle

\begin{abstract}
This document provides a first introduction into the typesetting system \LaTeX . After reading it, you will know everything there is to know about \LaTeX . 
\end{abstract}



% I add the table of contents only as a demonstration for your MSc thesis. In your articles for RS, you should not have a table of contents
\tableofcontents

\section{Introduction}


This is my \textbf{first} \emph{sentence}.

This is my \textit{second} sentence. You see that if you do a line break in the LaTeX source, signaling a paragraph, the first line of new paragraphs is indented by default. You could change this, but for the moment just leave it like that. 

This is my third paragraph. You can also do an explicit line break, and additionally put some distance between lines.\\[7mm]
Whether or not this is indented depends on whether you leave an empty line or not. 


\section{Doing equations}\label{sec: equations}

\subsection{Inline equations}
This is our first inline equation $\alpha = 3.5$. $\lambda = \sqrt{\alpha}$. 

\subsection{Numbered equations}

Test test

\begin{equation}
m = \left( \frac{a \cdot b}{c} \right) 
\end{equation}
%
where a = 5

\begin{equation}\label{eq: definition of lambda}
\lambda = \int_{m_0}^\infty f(\Theta) d \Theta
\end{equation}


\section{Labels and referencing}

In LaTeX, everything can be labeled, and after that it can be referenced. Consider for example section~\ref{sec: equations}, where eq.~\ref{eq: definition of lambda} are defined. You can also see this in fig.~\ref{fig: einstein}.

\section{Including figures}

Figures will not always be placed exactly where you put them - LaTeX is looking for an empty space, with certain preferences that you could modify. However, problems with figure placing usually disappear once you have more text. Hence, please only worry about figure placing if you have finished all your text!


\begin{figure}
\centering
\includegraphics[width=6cm]{einstein} % without file extension
\caption{This shows Einstein in his study}\label{fig: einstein}
\end{figure}  

\begin{figure}
\centering
\includegraphics[width=3cm]{einstein}
\includegraphics[width=3cm]{einstein}
\caption{This shows Einstein in his study}\label{fig: einstein2}
\end{figure}  



\begin{figure}
\centering
\includegraphics[width=6cm]{boxplot} % without file extension
\caption{This shows a boxplot}\label{fig: boxplot}
\end{figure} 

\section{Tables}

\subsection{Simple table}
\begin{tabular}{|c|c|}
\hline 
1 & 1 \\ 
\hline 
2 & 1 \\ 
\hline 
\end{tabular} 



\subsection{Proper table}



\begin{table*}\label{Table: Fit types}
  \centering
  \begin{tabular}{l@{\hspace{0.2cm}}l@{\hspace{0.2cm}}l} \toprule
  \textsc{Case} & \textsc{Explanation} & \textsc{Dimensions} \\ \midrule \addlinespace[0.2cm] 
  \multicolumn{3}{l}{Parameterization to virtual data, 3 PFTs:}  \\
  $V1$ & Data: SDD, GRO, reduced parameters &  $12,96$ \\ 
  $V2$ & Data: SDD, GRO, full parameters &  $26,96$  \\ 
  $V3$ & Data: SDD, reduced parameters &  $12,48$ \\ 
  $V4$ & Data: total SDD, reduced parameters &  $12,16$ \\ 
  $V5$ & Data: BM, reduced parameters &  $12,3$ \\  [0.2cm] 

  \multicolumn{3}{l}{Parameterization to Ecuadorian field data, 7 PFTs:} \\
  $E1$ & Data: SSD  &  $18,112$ \\ \bottomrule \\
\end{tabular}
\caption{Table caption}
\end{table*}


\section{References}

Referenced are extremely important \citep[see also][for more references]{Gintis-Costlysignalingand-2001,Archetti-Economicgametheory-2011}. However \citet{Cooper-CommunicationInCoordination-1992} note that this is not the case. 


% this is the style file. If you need to change something, google if the file you need is already there. If not (very uncommon) google makebst.
\bibliographystyle{chicago} 

% this is the bibtex libary file.
\bibliography{yourbibtexfile}

% Note: all files can be anywhere, just give the full path.


\end{document}
